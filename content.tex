\documentclass[a4paper,12pt]{article}
\usepackage{CJK}
\begin{document}

\begin{CJK}{UTF8}{gkai}
%\begin{CJK}{UTF8}{gbsn}
\section{密码电路故障攻击理论与技术}
\subsection{故障攻击的理论基础:包括攻击原理、基本假设、模型、方法和评价标准。此外,还将阐述一些常用的DFA和其他一些safe error的故障分析方法}
\subsubsection{故障分析的攻击原理和基本攻击方法,包括DFA和IFA,FSA等}
\subsubsection{故障模型分类和介绍}
\subsubsection{故障攻击的评价指标,包括所用故障模型,故障数,攻击轮数,攻击的时间复杂度等。}
\subsection{先进分析方法:包括一些当前研究热点,如故障和功耗攻击结合的分析方法}
\subsubsection{故障和传统密码学分析方法的结合}
\subsubsection{故障分析和功耗等旁路分析方法的结合}
\subsubsection{故障攻击在未知但可区分的故障模型下的应用}
\subsection{各类密码结构/算法的故障分析方法:针对常用的对称密码和非对称密码,给出其结构弱点和故障攻击分析方式}
\subsubsection{DES故障攻击}
\subsubsection{AES故障攻击}
\subsubsection{RSA故障攻击}
\subsubsection{ECC故障攻击}
\subsubsection{其他算法包括轻量级算法的故障分析}
\subsubsection{流密码和哈希函数的故障攻击简介}
\subsection{故障攻击实验环境:包括已有的故障攻击实施工具和攻击过程,如电压和时钟Glitch,激光和电磁辐射等}
\subsubsection{电压和时钟的故障实验环境和注入结果}
\subsubsection{电磁辐射的故障实验环境和注入结果}
\subsubsection{激光的故障实验环境和注入结果}
\subsubsection{多点故障注入的实验环境和实验结果}
\subsubsection{抗故障攻击的防御方法:已有的抗故障攻击防御技术}
\subsubsection{冗余计算的防护方案}
\subsubsection{掩码防护方案}
\subsubsection{抗故障攻击的电路单元}
\end{CJK}


\end{document}